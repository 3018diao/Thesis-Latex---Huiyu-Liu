\begin{abstract}
Automotive bus technology solves the problem of data interaction between each controller. With the implementation of CAN technology in automobiles, the huge amount of data collected from automobiles can be further processed, analyzed and beneficial for a wide range of problems and optimization.

These massive data have polytechnic value to enterprises. This thesis will focus on the CAN bus as a way to first describe what the characteristics of these data are and how they can be stored in the case of big data. The collected CAN bus data will be parsed and written to the database for easy searching by engineers. Different databases have various read and write as well as compression performance in the case of massive data. A performance comparison test of MySQL, ClickHouse, Cassandra and InfluxDB databases will be conducted to conclude which database is up to the task. Finally, the reasons for performance differences between different databases are explained in terms of database data structures.

This thesis is not only applicable to the field of automotive bus data, but also provides directions for data storage in the Internet of Things (IoT) and Industrial Internet.

\end{abstract}